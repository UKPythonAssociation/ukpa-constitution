%!TEX root = constitution.tex
\section{Amendment of Constitution}\label{sec:constitution}
As provided by clauses 224--227 of the Charities Act 2011:

    \subsection{}
    This constitution can only be amended:
    \begin{enumerate}
        \item by resolution agreed in writing by all members of \shortname{}; or
        \item by a resolution passed by a 75\% majority of votes cast at a general meeting of the members of \shortname{}.
    \end{enumerate}

    \subsection{}
    Any alteration of clause~\ref{sec:objects}, clause~\ref{sec:dissolution}, this clause, or of any provision where the alteration would provide authorisation for any benefit to be obtained by charity trustees or members of \shortname{} or persons connected with them, requires the prior written consent of the Charity Commission.

    \subsection{}
    No amendment that is inconsistent with the provisions of the Charities Act 2011 or the General Regulations shall be valid.

    \subsection{}
    A copy of any resolution altering the constitution, together with a copy of \shortname{}’s constitution as amended, must be sent to the Commission within 15 days from the date on which the resolution is passed. The amendment does not take effect until it has been recorded in the Register of Charities.

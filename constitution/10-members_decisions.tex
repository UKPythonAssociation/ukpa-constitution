%!TEX root = constitution.tex
\section{Members Decisions}\label{sec:member_decisions}

    \subsection{General Provisions}\label{sec:decisions_general}
    Except for those decisions that must be taken in a particular way as indicated in clause~\ref{sec:decisions_particular}, decisions of the members of \shortname{} may be taken either by vote at a general meeting as provided in clause~\ref{sec:decisions_vote} or by written resolution as provided in clause~\ref{sec:written_resolutions}.

    \subsection{Taking Ordinary Decisions by Vote}\label{sec:decisions_vote}
    Subject to clause~\ref{sec:decisions_particular}, any decision of the members of \shortname{} may be taken by means of a resolution at a general meeting. Such a resolution may be passed by a simple majority of votes cast at the meeting.

    \subsection{Taking Ordinary Decisions by Written Resolution without a General Meeting}\label{sec:written_resolutions}

        \subsubsection{}\label{sec:resolution_validity}
        Subject to clause~\ref{sec:decisions_particular}, a resolution in writing agreed by a simple majority of all the members who would have been entitled to vote upon it had it been proposed at a general meeting shall be effective, provided that:
        \begin{enumerate}
            \item a copy of the proposed resolution has been sent to all the members eligible to vote; and
            \item a simple majority of members has signified its agreement to the resolution in a document or documents which are received at the principal office within the period of 28 days beginning with the circulation date. The document signifying a member's agreement must be authenticated by their signature (or in the case of an organisation which is a member, by execution according to its usual procedure), by a statement of their identity accompanying the document, or in such other manner as \shortname{} has specified.
        \end{enumerate}

        \subsubsection{}
        The resolution in writing may comprise several copies to which one or more members has signified their agreement.

        \subsubsection{}\label{sec:voting_eligibility}
        Eligibility to vote on the resolution is limited to members who are members of \shortname{} on the date when the proposal is first circulated in accordance with clause~\ref{sec:resolution_validity}.

        \subsubsection{}
        Not less than 10\% of the members of \shortname{} may request the charity trustees to make a proposal for decision by the members.

        \subsubsection{}
        The charity trustees must within 21 days of receiving such a request comply with it if:
        \begin{enumerate}
            \item The proposal is not frivolous or vexatious, and does not involve the publication of defamatory material;
            \item The proposal is stated with sufficient clarity to enable effect to be given to it if it is agreed by the members; and
            \item Effect can lawfully be given to the proposal if it is so agreed.
        \end{enumerate}

        \subsubsection{}
        Clauses~\ref{sec:resolution_validity} to~\ref{sec:voting_eligibility} apply to a proposal made at the request of members.


    \subsection{Decisions that must be taken in a Particular Way}\label{sec:decisions_particular}

        \subsubsection{}
        Any decision to remove a trustee must be taken in accordance with clause~\ref{sec:removal}

        \subsubsection{}
        Any decision to amend this constitution must be taken in accordance with clause~\ref{sec:constitution}.

        \subsubsection{}
        Any decision to wind up or dissolve \shortname{} must be taken in accordance with clause~\ref{sec:dissolution}. Any decision to amalgamate or transfer the undertaking of \shortname{} to one or more other CIOs must be taken in accordance with the provisions of the Charities Act 2011.


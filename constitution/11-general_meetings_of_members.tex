%!TEX root = constitution.tex
\section{General Meetings of Members}\label{sec:general_meetings}

    \subsection{Types of General Meetings}\label{sec:general_meeting_types}
    There must be an annual general meeting (AGM) of the members of \shortname{}. The  first AGM must be held within 18 months of the registration of \shortname{}, and subsequent AGMs must be held at intervals of not more than 15 months. The AGM must receive the annual statement of accounts (duly audited or examined where applicable) and the trustees' annual report, and must elect trustees as required under clause~\ref{sec:appointment}.

    Other general meetings of the members of \shortname{} may be held at any time.

    All general meetings must be held in accordance with the following provisions.

    \subsection{Calling General Meetings}\label{sec:calling_general_meetings}

        \subsubsection{}
        The charity trustees:
        \begin{enumerate}
            \item must call the annual general meeting of the members of \shortname{} in accordance with clause~\ref{sec:general_meeting_types}, and identify it as such in the notice of the meeting; and
            \item may call any other general meeting of the members at any time.
        \end{enumerate}

        \subsubsection{}\label{sec:call_meeting_if}
        The charity trustees must, within 21 days, call a general meeting of the members of \shortname{} if:
        \begin{enumerate}
            \item\label{item:min_request} they receive a request to do so from at least 10\% of the members of \shortname{}; and

            \item the request states the general nature of the business to be dealt with at the meeting, and is authenticated by the member(s) making the request.
        \end{enumerate}

        \subsubsection{}
        If, at the time of any such request, there has not been any general meeting of the members of \shortname{} for more than 12 months, then item~\ref{item:min_request} of clause~\ref{sec:call_meeting_if} shall have effect as if 5\% were substituted for 10\%.

        \subsubsection{}
        Any such request may include particulars of a resolution that may properly be proposed, and is intended to be proposed, at the meeting.

        \subsubsection{}
        A resolution may only properly be proposed if it is lawful, and is not defamatory, frivolous or vexatious.

        \subsubsection{}
        Any general meeting called by the charity trustees at the request of the members of \shortname{} must be held within 28 days from the date on which it is called.

        \subsubsection{}
        If the charity trustees fail to comply with this obligation to call a general meeting at the request of its members, then the members who requested the meeting may themselves call a general meeting.

        \subsubsection{}
        A general meeting called in this way must be held not more than 3 months after the date when the members first requested the meeting.

        \subsubsection{}
        \shortname{} must reimburse any reasonable expenses incurred by the members calling a general meeting by reason of the failure of the charity trustees to duly call the meeting, but \shortname{} shall be entitled to be indemnified by the charity trustees who were responsible for such failure.

    \subsection{Notice of General Meetings}\label{sec:general_meetings_notice}

        \subsubsection{}\label{sec:general_meetings_notice_general}
        The charity trustees, or, as the case may be, the relevant members of \shortname{}, must give at least 14 clear days notice of any general meeting to all of the members, and to any charity trustee of \shortname{} who is not a member.

        \subsubsection{}
        If it is agreed by not less than 90\% of all members of \shortname{}, any resolution may be proposed and passed at the meeting even though the requirements of clause~\ref{sec:general_meetings_notice_general} have not been met. This clause does not apply where a specified period of notice is strictly required by another clause in this constitution, by the Charities Act 2011 or by the General Regulations.

        \subsubsection{}
        The notice of any general meeting must:
        \begin{enumerate}
            \item state the time and date of the meeting:
            \item give the address at which the meeting is to take place;
            \item give particulars of any resolution which is to be moved at the meeting, and of the general nature of any other business to be dealt with at the meeting; and
            \item if a proposal to alter the constitution of \shortname{} is to be considered at the meeting, include the text of the proposed alteration;
            \item include, with the notice for the AGM, the annual statement of accounts and trustees’ annual report, details of persons standing for election or re-election as trustee, or where allowed under clause~\ref{sec:comms}, details of where the information may be found on \shortname{}’s website.
        \end{enumerate}

        \subsubsection{}
        Proof that an envelope containing a notice was properly addressed, prepaid and posted; or that an electronic form of notice was properly addressed and sent, shall be conclusive evidence that the notice was given. Notice shall be deemed to be given 48 hours after it was posted or sent.

        \subsubsection{}
        The proceedings of a meeting shall not be invalidated because a member who was entitled to receive notice of the meeting did not receive it because of accidental omission by \shortname{}.

    \subsection{Chairing of General Meetings}\label{sec:chairing_general_meetings}
    The person nominated as chair by the charity trustees under clause~\ref{sec:chairing_of_meetings}, shall, if present at the general meeting and willing to act, preside as chair of the meeting. Subject to that, the members of \shortname{} who are present at a general meeting shall elect a chair to preside at the meeting.

    \subsection{Quorum at General Meetings}\label{sec:quorum_general_meetings}

        \subsubsection{}
        No business may be transacted at any general meeting of the members of \shortname{} unless a quorum is present when the meeting starts.

        \subsubsection{}
        Subject to the following provisions, the quorum for general meetings shall be the greater of 5\% or three members. An organisation represented by a person present at the meeting in accordance with clause~\ref{sec:representation_general_meetings}, is counted as being present in person.

        \subsubsection{}
        If the meeting has been called by or at the request of the members and a quorum is not present within 15 minutes of the starting time specified in the notice of the meeting, the meeting is closed.

        \subsubsection{}
        If the meeting has been called in any other way and a quorum is not present within 15 minutes of the starting time specified in the notice of the meeting, the chair must adjourn the meeting. The date, time and place at which the meeting will resume must either be announced by the chair or be notified to \shortname{}'s members at least seven clear days before the date on which it will resume.

        \subsubsection{}
        If a quorum is not present within 15 minutes of the start time of the adjourned meeting, the member or members present at the meeting constitute a quorum.

        \subsubsection{}
        If at any time during the meeting a quorum ceases to be present, the meeting may discuss issues and make recommendations to the trustees but may not make any decisions. If decisions are required which must be made by a meeting of the members, the meeting must be adjourned.

    \subsection{Voting at General Meetings}\label{sec:voting_general_meetings}

        \subsubsection{}
        Any decision other than one falling within clause~\ref{sec:decisions_particular} shall be taken by a simple majority of votes cast at the meeting. Every member has one vote unless otherwise provided in the rights of a particular class of membership under this constitution.

        \subsubsection{}
        A resolution put to the vote of a meeting shall be decided on a show of hands, unless (before or on the declaration of the result of the show of hands) a poll is duly demanded. A poll may be demanded by the chair or by at least 10\% of the members present in person or by proxy at the meeting.

        \subsubsection{}
        A poll demanded on the election of a person to chair the meeting or on a question of adjournment must be taken immediately. A poll on any other matter shall be taken, and the result of the poll shall be announced, in such manner as the chair of the meeting shall decide, provided that the poll must be taken, and the result of the poll announced, within 30 days of the demand for the poll.

        \subsubsection{}
        A poll may be taken:
        \begin{enumerate}
            \item at the meeting at which it was demanded; or
            \item at some other time and place specified by the chair; or
            \item through the use of postal or electronic communications.
        \end{enumerate}

        \subsubsection{}
        In the event of an equality of votes, whether on a show of hands or on a poll, the chair of the meeting shall have a second, or casting vote.

        \subsubsection{}
        Any objection to the qualification of any voter must be raised at the meeting at which the vote is cast and the decision of the chair of the meeting shall be final.

    \subsection{Representation of Organisations and Corporate Members}\label{sec:representation_general_meetings}
    An organisation or a corporate body that is a member of \shortname{} may, in accordance with its usual decision-making process, authorise a person to act as its representative at any general meeting of \shortname{}.

    The representative is entitled to exercise the same powers on behalf of the organisation or corporate body as the organisation or corporate body could exercise as an individual member of \shortname{}.

    \subsection{Adjournment of Meetings}\label{sec:adjournment_general_meetings}
    The chair may with the consent of a meeting at which a quorum is present (and shall if so directed by the meeting) adjourn the meeting to another time and/or place. No business may be transacted at an adjourned meeting except business which could properly have been transacted at the original meeting.
